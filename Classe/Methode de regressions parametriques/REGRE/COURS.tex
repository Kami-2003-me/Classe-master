\documentclass{article}
\usepackage{amsmath, amssymb, graphicx}
\usepackage{hyperref}

\title{R\'egression Lin\'eaire en R}
\author{}
\date{}

\begin{document}

\maketitle

\section{Introduction}
La r\'egression lin\'eaire est une m\'ethode statistique permettant de mod\'eliser la relation entre une variable d\'ependante ($Y$) et une ou plusieurs variables explicatives ($X$). Elle repose sur plusieurs hypoth\`eses fondamentales et permet d'effectuer des pr\'edictions et d'analyser l'influence des variables explicatives sur la variable cible.

Il existe deux types principaux de r\'egression lin\'eaire :
\begin{itemize}
    \item \textbf{R\'egression lin\'eaire simple} : Une seule variable explicative ($X$).
    \item \textbf{R\'egression lin\'eaire multiple} : Plusieurs variables explicatives ($X_1, X_2, ..., X_p$).
\end{itemize}

Dans ce cours, nous allons utiliser \textbf{R} pour illustrer ces concepts avec un exemple appliqu\'e en agriculture.

\section{R\'egression Lin\'eaire Simple}
\subsection{Importation des donn\'ees}
Nous utilisons un fichier de donn\'ees contenant des observations sur le rendement agricole ($Y$) en fonction de la quantit\'e d'engrais appliqu\'ee ($X$).

\begin{verbatim}
# Importation des donn\'ees
RLS <- read.table(file.choose(), header=TRUE)
attach(RLS)
names(RLS)  # Afficher les noms des variables
\end{verbatim}

\subsection{Visualisation des donn\'ees}
Avant d'ajuster le mod\`ele, nous affichons un nuage de points pour observer la relation entre les variables :

\begin{verbatim}
plot(Rendement ~ Engrais, xlab="Engrais (Kg/ha)", ylab="Rendement (t/ha)")
\end{verbatim}

\subsection{Ajustement du mod\`ele}
Nous utilisons la fonction \texttt{lm()} pour ajuster une r\'egression lin\'eaire simple :

\begin{verbatim}
# Estimation des param\`etres du mod\`ele
modele1 <- lm(Rendement ~ Engrais)
summary(modele1)  # Afficher les r\'esultats du mod\`ele
\end{verbatim}

\subsection{Visualisation de la droite de r\'egression}
Nous tra\c cons la droite ajust\'ee sur le nuage de points :

\begin{verbatim}
plot(Rendement ~ Engrais, xlab="Engrais (Kg/ha)", ylab="Rendement (t/ha)")
abline(modele1, col="blue")
\end{verbatim}

\subsection{Validation du Mod\`ele}
\subsubsection{Test de normalit\'e des r\'esidus}
\begin{verbatim}
par(mfrow=c(1,2))
hist(residuals(modele1), main="Histogramme des r\'esidus")
qqnorm(resid(modele1))
qqline(resid(modele1), lty=2, col="blue")
shapiro.test(resid(modele1))  # Test de Shapiro-Wilk
\end{verbatim}

\subsubsection{Test de l'homog\'en\'eit\'e des r\'esidus}
\begin{verbatim}
library(lmtest)
plot(residuals(modele1) ~ fitted(modele1), xlab="Valeurs pr\'edites", ylab="R\'esidus")
bptest(modele1)  # Test de Breusch-Pagan
\end{verbatim}

\subsubsection{Test d'autocorr\'elation des r\'esidus}
\begin{verbatim}
library(lmtest)
dwtest(modele1)  # Test de Durbin-Watson
\end{verbatim}

\section{R\'egression Lin\'eaire Multiple}
\subsection{Importation et Visualisation des donn\'ees}
Nous incluons ici plusieurs variables explicatives ($Engrais, Irrigation, Type\_de\_sol$).

\begin{verbatim}
RLM <- read.table(file.choose(), header=TRUE)
attach(RLM)
pairs(RLM)  # Matrice de dispersion
\end{verbatim}

\subsection{Ajustement du Mod\`ele}

\begin{verbatim}
modele2 <- lm(Rendement ~ Engrais + Irrigation + Type_de_sol)
summary(modele2)
\end{verbatim}

\subsection{Validation du Mod\`ele}
\subsubsection{Test de normalit\'e des r\'esidus}
\begin{verbatim}
par(mfrow=c(1,2))
hist(residuals(modele2), main="Histogramme des r\'esidus")
qqnorm(resid(modele2))
qqline(resid(modele2), lty=2, col="blue")
shapiro.test(resid(modele2))
\end{verbatim}

\subsubsection{Test de l'homog\'en\'eit\'e des r\'esidus}
\begin{verbatim}
plot(residuals(modele2) ~ fitted(modele2), xlab="Valeurs pr\'edites", ylab="R\'esidus")
bptest(modele2)
\end{verbatim}

\subsubsection{Test de multicolin\'earit\'e}
\begin{verbatim}
library(car)
vif(modele2)  # Facteur d'inflation de la variance
\end{verbatim}

\subsection{Conclusion}
La r\'egression lin\'eaire est un outil puissant pour analyser les relations entre variables quantitatives. Les tests de validation sont essentiels pour s'assurer que les hypoth\`eses du mod\`ele sont respect\'ees.

\textbf{Extensions possibles} :
\begin{itemize}
    \item R\'egression polynomiale
    \item R\'egression logistique
    \item R\'egression ridge et lasso
\end{itemize}

Ces techniques permettent d'affiner les mod\`eles et de mieux capturer les relations complexes entre les variables.

\end{document}

