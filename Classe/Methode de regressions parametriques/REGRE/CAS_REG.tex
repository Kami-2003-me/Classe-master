\documentclass[a4paper,12pt]{article}
\usepackage{graphicx}
\usepackage{amsmath, amssymb}
\usepackage{booktabs}
\usepackage{hyperref}
\usepackage{xcolor}
\usepackage{listings}


% Définition du style pour le code R
\lstdefinestyle{Rstyle}{
    language=R,
    basicstyle=\ttfamily\footnotesize,
    keywordstyle=\color{blue},
    commentstyle=\color{green},
    stringstyle=\color{red},
    breaklines=true,
    frame=single,
    captionpos=b
}

\title{\textbf{Impact des valeurs aberrantes sur une régression linéaire}}
\author{Université XYZ}
\date{\today}

\begin{document}

\maketitle

\section{Introduction}
Les valeurs aberrantes sont des observations qui s'écartent fortement des autres données. Elles peuvent fausser les estimations des paramètres et conduire à des erreurs de prédiction importantes. Cette étude examine leur impact sur une régression linéaire expliquant le \textit{Salaire} en fonction de l'\textit{Expérience} et de l'\textit{Éducation}.

\section{Présentation des données}
Nous utilisons un jeu de données hypothétique contenant 500 observations avec les variables suivantes :
\begin{itemize}
    \item \textbf{Salaire} : revenu annuel en milliers d'euros.
    \item \textbf{Expérience} : nombre d'années d'expérience.
    \item \textbf{Éducation} : nombre d'années d'études après le secondaire.
\end{itemize}

Exemple de données :
\begin{center}
    \begin{tabular}{ccc}
    \toprule
    Salaire (k€) & Expérience (années) & Éducation (années) \\
    \midrule
    25  & 2  & 5  \\
    40  & 5  & 6  \\
    120 & 15 & 10 \\
    \textbf{300} & \textbf{4} & \textbf{3} \\
    35  & 3  & 5  \\
    \bottomrule
    \end{tabular}
\end{center}

\section{Détection des valeurs aberrantes}

\subsection{Méthodes graphiques}
\begin{itemize}
    \item \textbf{Boxplot} : permet d'identifier visuellement les valeurs extrêmes.
    \item \textbf{Diagramme de dispersion} : met en évidence les observations éloignées.
\end{itemize}

\subsection{Méthodes statistiques}
\begin{itemize}
    \item \textbf{Distance interquartile (IQR)} :
    \begin{equation}
        \text{Valeurs aberrantes} \in [Q1 - 1.5 IQR, Q3 + 1.5 IQR]
    \end{equation}
    \item \textbf{Z-score} : une observation est aberrante si son Z-score est supérieur à 3.
\end{itemize}

\section{Impact sur la régression}
Nous estimons le modèle suivant :
\begin{equation}
    Salaire_i = \beta_0 + \beta_1 Experience_i + \beta_2 Education_i + \varepsilon_i
\end{equation}

\subsection{Comparaison des méthodes de traitement}
\begin{center}
    \begin{tabular}{lccc}
    \toprule
    Méthode & Effet sur les coefficients & Avantages & Inconvénients \\
    \midrule
    Suppression des outliers & Coefficients plus stables & Facile à interpréter & Perte d'information \\
    Transformation logarithmique & Réduction des valeurs extrêmes & Utile pour asymétrie & Moins efficace pour grandes valeurs \\
    Régression robuste & Estimations plus fiables & Gère bien les outliers & Plus complexe à implémenter \\
    \bottomrule
    \end{tabular}
\end{center}

\section{Application en R}
Le code suivant permet de générer les données et d'appliquer différentes méthodes de détection et de correction des valeurs aberrantes.

\subsection{Chargement des bibliothèques et génération des données}
\begin{lstlisting}[style=Rstyle, caption=Génération des données]
install.packages(c("ggplot2", "dplyr", "MASS", "car"))
library(ggplot2)
library(dplyr)
library(MASS)
library(car)

set.seed(123)
n <- 500
data <- data.frame(
  Experience = runif(n, 1, 20),
  Education = sample(3:10, n, replace = TRUE),
  Salaire = 25 + 3 * runif(n, 1, 20) + 2 * sample(3:10, n, replace = TRUE) + rnorm(n, 0, 5)
)
data$Salaire[c(10, 50, 100)] <- c(150, 300, 500)
head(data)
\end{lstlisting}

\subsection{Détection des valeurs aberrantes}
\begin{lstlisting}[style=Rstyle, caption=Détection des valeurs aberrantes]
Q1 <- quantile(data$Salaire, 0.25)
Q3 <- quantile(data$Salaire, 0.75)
IQR <- Q3 - Q1
seuil_inf <- Q1 - 1.5 * IQR
seuil_sup <- Q3 + 1.5 * IQR
data$outlier_iqr <- ifelse(data$Salaire < seuil_inf | data$Salaire > seuil_sup, TRUE, FALSE)
\end{lstlisting}

\subsection{Régression linéaire et robuste}
\begin{lstlisting}[style=Rstyle, caption=Régression et impact des outliers]
model1 <- lm(Salaire ~ Experience + Education, data = data)
summary(model1)
data_clean <- data[!data$outlier_iqr, ]
model2 <- lm(Salaire ~ Experience + Education, data = data_clean)
summary(model2)
model3 <- rlm(Salaire ~ Experience + Education, data = data)
summary(model3)
\end{lstlisting}

\section{Conclusion}
Les valeurs aberrantes faussent les estimations et augmentent les erreurs de prédiction. La suppression ou l'utilisation de méthodes robustes permet d'améliorer la fiabilité du modèle. 

\end{document}
